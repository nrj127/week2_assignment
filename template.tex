%% bare_conf.tex
%% V1.4
%% 2012/12/27
%% by Michael Shell
%% See:
%% http://www.michaelshell.org/
%% for current contact information.
%%
%% This is a skeleton file demonstrating the use of IEEEtran.cls
%% (requires IEEEtran.cls version 1.8 or later) with an IEEE conference paper.
%%
%% Support sites:
%% http://www.michaelshell.org/tex/ieeetran/
%% http://www.ctan.org/tex-archive/macros/latex/contrib/IEEEtran/
%% and
%% http://www.ieee.org/

%%*************************************************************************
%% Legal Notice:
%% This code is offered as-is without any warranty either expressed or
%% implied; without even the implied warranty of MERCHANTABILITY or
%% FITNESS FOR A PARTICULAR PURPOSE! 
%% User assumes all risk.
%% In no event shall IEEE or any contributor to this code be liable for
%% any damages or losses, including, but not limited to, incidental,
%% consequential, or any other damages, resulting from the use or misuse
%% of any information contained here.
%%
%% All comments are the opinions of their respective authors and are not
%% necessarily endorsed by the IEEE.
%%
%% This work is distributed under the LaTeX Project Public License (LPPL)
%% ( http://www.latex-project.org/ ) version 1.3, and may be freely used,
%% distributed and modified. A copy of the LPPL, version 1.3, is included
%% in the base LaTeX documentation of all distributions of LaTeX released
%% 2003/12/01 or later.
%% Retain all contribution notices and credits.
%% ** Modified files should be clearly indicated as such, including  **
%% ** renaming them and changing author support contact information. **
%%
%% File list of work: IEEEtran.cls, IEEEtran_HOWTO.pdf, bare_adv.tex,
%%                    bare_conf.tex, bare_jrnl.tex, bare_jrnl_compsoc.tex,
%%                    bare_jrnl_transmag.tex
%%*************************************************************************

% *** Authors should verify (and, if needed, correct) their LaTeX system  ***
% *** with the testflow diagnostic prior to trusting their LaTeX platform ***
% *** with production work. IEEE's font choices can trigger bugs that do  ***
% *** not appear when using other class files.                            ***
% The testflow support page is at:
% http://www.michaelshell.org/tex/testflow/



% Note that the a4paper option is mainly intended so that authors in
% countries using A4 can easily print to A4 and see how their papers will
% look in print - the typesetting of the document will not typically be
% affected with changes in paper size (but the bottom and side margins will).
% Use the testflow package mentioned above to verify correct handling of
% both paper sizes by the user's LaTeX system.
%
% Also note that the "draftcls" or "draftclsnofoot", not "draft", option
% should be used if it is desired that the figures are to be displayed in
% draft mode.
%
\documentclass[conference]{IEEEtran}
% Add the compsoc option for Computer Society conferences.
%
% If IEEEtran.cls has not been installed into the LaTeX system files,
% manually specify the path to it like:
% \documentclass[conference]{../sty/IEEEtran}



% Some very useful LaTeX packages include:
% (uncomment the ones you want to load)


% For directly writing german umlauts uncomment the appropriate line for
% your operating system:
% Windows:
% \usepackage[ansinew]{inputenc}
% Linux:
\usepackage[latin1]{inputenc}
% Mac
% \usepackage[applemac]{inputenc}
% If none of the above lines work you can also try the following:
% \usepackage[utf8]{inputenc}



% *** MISC UTILITY PACKAGES ***
%
%\usepackage{ifpdf}
% Heiko Oberdiek's ifpdf.sty is very useful if you need conditional
% compilation based on whether the output is pdf or dvi.
% usage:
% \ifpdf
%   % pdf code
% \else
%   % dvi code
% \fi
% The latest version of ifpdf.sty can be obtained from:
% http://www.ctan.org/tex-archive/macros/latex/contrib/oberdiek/
% Also, note that IEEEtran.cls V1.7 and later provides a builtin
% \ifCLASSINFOpdf conditional that works the same way.
% When switching from latex to pdflatex and vice-versa, the compiler may
% have to be run twice to clear warning/error messages.






% *** CITATION PACKAGES ***
%
\usepackage{cite}
% cite.sty was written by Donald Arseneau
% V1.6 and later of IEEEtran pre-defines the format of the cite.sty package
% \cite{} output to follow that of IEEE. Loading the cite package will
% result in citation numbers being automatically sorted and properly
% "compressed/ranged". e.g., [1], [9], [2], [7], [5], [6] without using
% cite.sty will become [1], [2], [5]--[7], [9] using cite.sty. cite.sty's
% \cite will automatically add leading space, if needed. Use cite.sty's
% noadjust option (cite.sty V3.8 and later) if you want to turn this off
% such as if a citation ever needs to be enclosed in parenthesis.
% cite.sty is already installed on most LaTeX systems. Be sure and use
% version 4.0 (2003-05-27) and later if using hyperref.sty. cite.sty does
% not currently provide for hyperlinked citations.
% The latest version can be obtained at:
% http://www.ctan.org/tex-archive/macros/latex/contrib/cite/
% The documentation is contained in the cite.sty file itself.






% *** GRAPHICS RELATED PACKAGES ***
%
\ifCLASSINFOpdf
  \usepackage[pdftex]{graphicx}
  % declare the path(s) where your graphic files are
  % \graphicspath{{../pdf/}{../jpeg/}}
  % and their extensions so you won't have to specify these with
  % every instance of \includegraphics
  % \DeclareGraphicsExtensions{.pdf,.jpeg,.png}
\else
  % or other class option (dvipsone, dvipdf, if not using dvips). graphicx
  % will default to the driver specified in the system graphics.cfg if no
  % driver is specified.
  \usepackage[dvips]{graphicx}
  % declare the path(s) where your graphic files are
  % \graphicspath{{../eps/}}
  % and their extensions so you won't have to specify these with
  % every instance of \includegraphics
  % \DeclareGraphicsExtensions{.eps}
\fi
% graphicx was written by David Carlisle and Sebastian Rahtz. It is
% required if you want graphics, photos, etc. graphicx.sty is already
% installed on most LaTeX systems. The latest version and documentation
% can be obtained at: 
% http://www.ctan.org/tex-archive/macros/latex/required/graphics/
% Another good source of documentation is "Using Imported Graphics in
% LaTeX2e" by Keith Reckdahl which can be found at:
% http://www.ctan.org/tex-archive/info/epslatex/
%
% latex, and pdflatex in dvi mode, support graphics in encapsulated
% postscript (.eps) format. pdflatex in pdf mode supports graphics
% in .pdf, .jpeg, .png and .mps (metapost) formats. Users should ensure
% that all non-photo figures use a vector format (.eps, .pdf, .mps) and
% not a bitmapped formats (.jpeg, .png). IEEE frowns on bitmapped formats
% which can result in "jaggedy"/blurry rendering of lines and letters as
% well as large increases in file sizes.
%
% You can find documentation about the pdfTeX application at:
% http://www.tug.org/applications/pdftex





% *** MATH PACKAGES ***
%
\usepackage[cmex10]{amsmath}
% A popular package from the American Mathematical Society that provides
% many useful and powerful commands for dealing with mathematics. If using
% it, be sure to load this package with the cmex10 option to ensure that
% only type 1 fonts will utilized at all point sizes. Without this option,
% it is possible that some math symbols, particularly those within
% footnotes, will be rendered in bitmap form which will result in a
% document that can not be IEEE Xplore compliant!
%
% Also, note that the amsmath package sets \interdisplaylinepenalty to 10000
% thus preventing page breaks from occurring within multiline equations. Use:
%\interdisplaylinepenalty=2500
% after loading amsmath to restore such page breaks as IEEEtran.cls normally
% does. amsmath.sty is already installed on most LaTeX systems. The latest
% version and documentation can be obtained at:
% http://www.ctan.org/tex-archive/macros/latex/required/amslatex/math/





% *** SPECIALIZED LIST PACKAGES ***
%
%\usepackage{algorithmic}
% algorithmic.sty was written by Peter Williams and Rogerio Brito.
% This package provides an algorithmic environment fo describing algorithms.
% You can use the algorithmic environment in-text or within a figure
% environment to provide for a floating algorithm. Do NOT use the algorithm
% floating environment provided by algorithm.sty (by the same authors) or
% algorithm2e.sty (by Christophe Fiorio) as IEEE does not use dedicated
% algorithm float types and packages that provide these will not provide
% correct IEEE style captions. The latest version and documentation of
% algorithmic.sty can be obtained at:
% http://www.ctan.org/tex-archive/macros/latex/contrib/algorithms/
% There is also a support site at:
% http://algorithms.berlios.de/index.html
% Also of interest may be the (relatively newer and more customizable)
% algorithmicx.sty package by Szasz Janos:
% http://www.ctan.org/tex-archive/macros/latex/contrib/algorithmicx/




% *** ALIGNMENT PACKAGES ***
%
\usepackage{array}
% Frank Mittelbach's and David Carlisle's array.sty patches and improves
% the standard LaTeX2e array and tabular environments to provide better
% appearance and additional user controls. As the default LaTeX2e table
% generation code is lacking to the point of almost being broken with
% respect to the quality of the end results, all users are strongly
% advised to use an enhanced (at the very least that provided by array.sty)
% set of table tools. array.sty is already installed on most systems. The
% latest version and documentation can be obtained at:
% http://www.ctan.org/tex-archive/macros/latex/required/tools/


% IEEEtran contains the IEEEeqnarray family of commands that can be used to
% generate multiline equations as well as matrices, tables, etc., of high
% quality.




% *** SUBFIGURE PACKAGES ***
\ifCLASSOPTIONcompsoc
  \usepackage[caption=false,font=normalsize,labelfont=sf,textfont=sf]{subfig}
\else
  \usepackage[caption=false,font=footnotesize]{subfig}
\fi
% subfig.sty, written by Steven Douglas Cochran, is the modern replacement
% for subfigure.sty, the latter of which is no longer maintained and is
% incompatible with some LaTeX packages including fixltx2e. However,
% subfig.sty requires and automatically loads Axel Sommerfeldt's caption.sty
% which will override IEEEtran.cls' handling of captions and this will result
% in non-IEEE style figure/table captions. To prevent this problem, be sure
% and invoke subfig.sty's "caption=false" package option (available since
% subfig.sty version 1.3, 2005/06/28) as this is will preserve IEEEtran.cls
% handling of captions.
% Note that the Computer Society format requires a larger sans serif font
% than the serif footnote size font used in traditional IEEE formatting
% and thus the need to invoke different subfig.sty package options depending
% on whether compsoc mode has been enabled.
%
% The latest version and documentation of subfig.sty can be obtained at:
% http://www.ctan.org/tex-archive/macros/latex/contrib/subfig/




% *** FLOAT PACKAGES ***
%
\usepackage{fixltx2e}
% fixltx2e, the successor to the earlier fix2col.sty, was written by
% Frank Mittelbach and David Carlisle. This package corrects a few problems
% in the LaTeX2e kernel, the most notable of which is that in current
% LaTeX2e releases, the ordering of single and double column floats is not
% guaranteed to be preserved. Thus, an unpatched LaTeX2e can allow a
% single column figure to be placed prior to an earlier double column
% figure. The latest version and documentation can be found at:
% http://www.ctan.org/tex-archive/macros/latex/base/


%\usepackage{stfloats}
% stfloats.sty was written by Sigitas Tolusis. This package gives LaTeX2e
% the ability to do double column floats at the bottom of the page as well
% as the top. (e.g., "\begin{figure*}[!b]" is not normally possible in
% LaTeX2e). It also provides a command:
%\fnbelowfloat
% to enable the placement of footnotes below bottom floats (the standard
% LaTeX2e kernel puts them above bottom floats). This is an invasive package
% which rewrites many portions of the LaTeX2e float routines. It may not work
% with other packages that modify the LaTeX2e float routines. The latest
% version and documentation can be obtained at:
% http://www.ctan.org/tex-archive/macros/latex/contrib/sttools/
% Do not use the stfloats baselinefloat ability as IEEE does not allow
% \baselineskip to stretch. Authors submitting work to the IEEE should note
% that IEEE rarely uses double column equations and that authors should try
% to avoid such use. Do not be tempted to use the cuted.sty or midfloat.sty
% packages (also by Sigitas Tolusis) as IEEE does not format its papers in
% such ways.
% Do not attempt to use stfloats with fixltx2e as they are incompatible.
% Instead, use Morten Hogholm'a dblfloatfix which combines the features
% of both fixltx2e and stfloats:
%
% \usepackage{dblfloatfix}
% The latest version can be found at:
% http://www.ctan.org/tex-archive/macros/latex/contrib/dblfloatfix/




% *** PDF, URL AND HYPERLINK PACKAGES ***
%
%\usepackage{url}
% url.sty was written by Donald Arseneau. It provides better support for
% handling and breaking URLs. url.sty is already installed on most LaTeX
% systems. The latest version and documentation can be obtained at:
% http://www.ctan.org/tex-archive/macros/latex/contrib/url/
% Basically, \url{my_url_here}.




% *** Do not adjust lengths that control margins, column widths, etc. ***
% *** Do not use packages that alter fonts (such as pslatex).         ***
% There should be no need to do such things with IEEEtran.cls V1.6 and later.
% (Unless specifically asked to do so by the journal or conference you plan
% to submit to, of course. )


% add custom packages
\usepackage{color}
\definecolor{tumblue}{rgb}{0, 0.4, 0.74}



% correct bad hyphenation here
\hyphenation{op-tical net-works semi-conduc-tor}


\begin{document}

% Add the seminar's cover page
\begin{figure*}[!h]

  \includegraphics{pics/IN.pdf} \hfill \includegraphics{pics/tumlogo.pdf}
 
  \vspace*{1cm}
  {\large \textsf{Fakult{\"a}t f{\"u}r Informatik}}\\
  {\large \textsf{Lehrstuhl f{\"u}r Bildverarbeitung und Mustererkennung}}\\
 

  \vspace*{5cm}
%
%
% TITEL DER ARBEIT
%
%
  {\color{tumblue} \Huge \bf \textsf{Image Classification using the deep learning caffe framework}}\\  % HIER EINSETZEN!

  \vspace*{1cm}
%
%
% NAME DES STUDENTEN (auf Titelblatt)
%
% 
  {\Large \bf \textsf{Neeraj Sujan - 03656452}}\\   
 % {\Large \bf \textsf{Github $:$ https:$//$github.com/nrj127/week2$_$assignment}}\\               %
  % HIER EINSETZEN!
  
  
 
  \vspace*{8cm}
  {\Large \textsf{Praktikum \emph{Maschninelles Lernen f{\"u}r Anwendung des Computersehens} SoSe2015}}\\
 
  \vspace*{1cm} 
  \begin{tabular}{ll}
%
%
% NAME DES BETREUERS
%
%
  %  {\Large \bf \textsf{Advisor:}} &
    %{\Large \textsf{Name of advisor}}\\                  % HIER EINSETZEN!
 %   \\

    %{\Large \bf \textsf{Supervisor:}} &
    %{\Large \textsf{Prof. Dr.-Ing Matthias Althoff}}\\
   % \\

%
%
% ABGABETERMIN
%
%
    {\Large \bf \textsf{Submission:}} &
    {\Large \textsf{08. May 2015}}

  \end{tabular}
  
\end{figure*}


%
% paper title
% can use linebreaks \\ within to get better formatting as desired
% Do not put math or special symbols in the title.
\title{Image Classification using the Caffe framework.}


% author names and affiliations
% use a multiple column layout for up to three different
% affiliations
%\author{\IEEEauthorblockN{Michael Shell}
%\IEEEauthorblockA{School of Electrical and\\Computer Engineering\\
%Georgia Institute of Technology\\
%Atlanta, Georgia 30332--0250\\
%Email: http://www.michaelshell.org/contact.html}
%\and
%\IEEEauthorblockN{Homer Simpson}
%\IEEEauthorblockA{Twentieth Century Fox\\
%Springfield, USA\\
%Email: homer@thesimpsons.com}
%\and
%\IEEEauthorblockN{James Kirk\\ and Montgomery Scott}
%\IEEEauthorblockA{Starfleet Academy\\
%San Francisco, California 96678-2391\\
%Telephone: (800) 555--1212\\
%Fax: (888) 555--1212}}

\author{ 
\IEEEauthorblockN{Neeraj Sujan}
\IEEEauthorblockA{Fakult{\"a}t f{\"u}r Informatik\\
Technische Universit{\"a}t M{\"u}nchen\\
%Atlanta, Georgia 30332--0250\\
Email: n.p.sujan@tum.de}
}
% conference papers do not typically use \thanks and this command
% is locked out in conference mode. If really needed, such as for
% the acknowledgment of grants, issue a \IEEEoverridecommandlockouts
% after \documentclass

% for over three affiliations, or if they all won't fit within the width
% of the page, use this alternative format:
% 
%\author{\IEEEauthorblockN{Michael Shell\IEEEauthorrefmark{1},
%Homer Simpson\IEEEauthorrefmark{2},
%James Kirk\IEEEauthorrefmark{3}, 
%Montgomery Scott\IEEEauthorrefmark{3} and
%Eldon Tyrell\IEEEauthorrefmark{4}}
%\IEEEauthorblockA{\IEEEauthorrefmark{1}School of Electrical and Computer Engineering\\
%Georgia Institute of Technology,
%Atlanta, Georgia 30332--0250\\ Email: see http://www.michaelshell.org/contact.html}
%\IEEEauthorblockA{\IEEEauthorrefmark{2}Twentieth Century Fox, Springfield, USA\\
%Email: homer@thesimpsons.com}
%\IEEEauthorblockA{\IEEEauthorrefmark{3}Starfleet Academy, San Francisco, California 96678-2391\\
%Telephone: (800) 555--1212, Fax: (888) 555--1212}
%\IEEEauthorblockA{\IEEEauthorrefmark{4}Tyrell Inc., 123 Replicant Street, Los Angeles, California 90210--4321}}




% use for special paper notices
%\IEEEspecialpapernotice{(Invited Paper)}




% make the title area
\maketitle

% As a general rule, do not put math, special symbols or citations
% in the abstract
\begin{abstract}
The report  discusses the popular caffe framework to classify images using deep learning algorithms.The training model is based on the  popular imagenet example and the data from the ilsvrc 2012 challenge \cite{IEEEhowto:kopka}. The images used for classification were provided by the chair of pattern recognition affiliated to the faculty of computer science, TU M{\"u}nchen.The paper also discusses the popular back propogation algorithm. 
\end{abstract}

% no keywords




% For peer review papers, you can put extra information on the cover
% page as needed:
% \ifCLASSOPTIONpeerreview
% \begin{center} \bfseries EDICS Category: 3-BBND \end{center}
% \fi
%
% For peerreview papers, this IEEEtran command inserts a page break and
% creates the second title. It will be ignored for other modes.
\IEEEpeerreviewmaketitle



\section{Introduction}
% no \IEEEPARstart

% You must have at least 2 lines in the paragraph with the drop letter
% (should never be an issue)

Neural Networks: Neural Networks or Artificial Neural Networks,as it is popularly known,  are a family of machine learning algorithms, that builds upon the popular single layer perceptron, developed by Rosenblatt. Neural Networks works on the principle of multiple hidden layer with input and output neurons. Each hidden layer contains an activation function, such as a sigmoidal function to make the network as efficient as possible. As in other classification algorithms, the goal of the network remains to minimize the error function.

\subsection{Caffe Framework}
The Caffe framework \cite{caffe} from UC Berkeley is designed to let researchers and novice learners create and explore convolutional neural networks  and other Deep Neural Networks easily.
Caffe provides state-of-the art modeling for advancing and deploying deep learning in research and industry with support for a wide variety of architectures and efficint implementations of prediction and learning.The Caffe framework follows a layered and a hierarchial  architecture

\subsubsection{Back Propogation Algorithm}
The following four steps describe the Backpropagation algorithm (taken from Bishop's Book)

1. Apply an input vector $ x _{n}$ to the network and forward propagate through the network using  

\begin{align}
 a _{j} = \sum\limits_{i} w _{ji} z _{i}  
\end{align}


\begin{align}
  z_{j} = h(a_{j})
\end{align}

to find the activation of all the hidden and output units. \\
2. Evaluate the $\delta _{k}$ for all the output units using

\begin{align}
\delta _{k} = y_{k} - t_{k}
\end{align}
\\
3. Backpropagate the $\delta^{'} s using $ 

\begin{align}
 \delta _{j} = h^{'}(a_{j}) \sum\limits_{k} w_{kj} \delta_{k}
\end{align}
to obtain $ \delta_{j}$ for each hidden unit in the network.\\
4. Use the below equation to evaluate the required derivatives

\begin{align}
\frac{\partial E_{n}}{\partial w_{ji}} = \delta_{j} z_{i}
\end{align}

Modification : If the input neuron is modified such that the activation function is not sigmoidal, then the following changes in the back-propagation algorithm will take place.

In equation (2),   $ z_{j} = h(a_{j}) = f( \sum\limits_{i} w _{ji} z _{i} + b) $. Equation (4) will change correspondingly to $\delta _{j} = f^{'}(\sum\limits_{i} w _{ji} z _{i} + b) \sum\limits_{k} w_{kj} \delta_{k} $ \\
 

\subsubsection{Experiment}


The figure (Figure 1) shows the 8 images that were to be classified using the ilsvrc2 2012 training model \cite{IEEEhowto:kopka}. The experiment involved the following steps

1. Installing the caffe framework and understanding its architecture. \\
2. Running  caffe installation and installing all the necesarry dependencies. \\
3. Using the  BVLC Reference CaffeNet provided\cite{caffe} as part of the installation for our training model. \\
4. Writing code to test the prediction, probability and entropy of the given 8 images.\\
5. Pre-processing the image to a obtain a better performance.\\


\begin{figure*}[!t]
\centering
\subfloat[Image1]{\includegraphics[width=0.5in]{ex2_images/IMG_1.jpg}%
\label{fig_first_case}} 
\hfil
\subfloat[Image2]{\includegraphics[width=0.5in]{ex2_images/IMG_2.jpg}%
\label{fig_second_case}}
\hfil
\subfloat[Image3]{\includegraphics[width=0.5in]{ex2_images/IMG_3.jpg}%
\label{fig_first_case}}
\hfil
\subfloat[Image4]{\includegraphics[width=0.5in]{ex2_images/IMG_4.jpg}%
\label{fig_first_case}} \\
%\hfil
\subfloat[Image5]{\includegraphics[width=0.5in]{ex2_images/IMG_5.jpg}%
\label{fig_first_case}}
\hfil
\subfloat[Image6]{\includegraphics[width=0.5in]{ex2_images/IMG_6.jpg}%
\label{fig_first_case}}
\hfil
\subfloat[Image7]{\includegraphics[width=0.5in]{ex2_images/IMG_7.jpg}%
\label{fig_first_case}}
\hfil
\subfloat[Image8]{\includegraphics[width=0.5in]{ex2_images/IMG_8.jpg}%
\label{fig_first_case}}
\hfil


\caption{Images provided}
\label{fig_sim}

\end{figure*}



Probability of class \cite {probability distribution} : In a multi-output classification problem, the probability if a class plays a very import role. The probability of a class indicates the likelihood of occurrence of a particular class. Given a particular unknown input, the corresponding output vector O, then the estimated probabilty that it belongs to each class is given as follows.

\[P_{cO}(c \mid O) = \frac{p(c \mid O)}{\Sigma _{c^{'} } p(c^{l} \mid O )  }\]


Entropy : Entropy is defined as the measure of impurity. For a binary class, it is represented as follows

\[Entropy = - p(a) *log(p(a)) - p(b)*log(p(b))\].



%\begin{figure}[h]
%\begin{center}
%\includegraphics{pics/grins.pdf}
%\end{center}
%\caption{A vector graphic loaded from a PDF file}
%\label{Pic1}
%\end{figure}

%\begin{figure}[h]
%\begin{center}
%\includegraphics{pics/grins.png}
%\end{center}
%\caption{A bitmap graphic loaded from a PNG file}
%\label{Pic2}
%\end{figure}


% An example of a floating figure using the graphicx package.
% Note that \label must occur AFTER (or within) \caption.
% For figures, \caption should occur after the \includegraphics.
% Note that IEEEtran v1.7 and later has special internal code that
% is designed to preserve the operation of \label within \caption
% even when the captionsoff option is in effect. However, because
% of issues like this, it may be the safest practice to put all your
% \label just after \caption rather than within \caption{}.
%
% Reminder: the "draftcls" or "draftclsnofoot", not "draft", class
% option should be used if it is desired that the figures are to be
% displayed while in draft mode.
%
%\begin{figure}[!t]
%\centering
%\includegraphics[width=2.5in]{myfigure}
% where an .eps filename suffix will be assumed under latex, 
% and a .pdf suffix will be assumed for pdflatex; or what has been declared
% via \DeclareGraphicsExtensions.
%\caption{Simulation Results.}
%\label{fig_sim}
%\end{figure}

% Note that IEEE typically puts floats only at the top, even when this
% results in a large percentage of a column being occupied by floats.


% An example of a double column floating figure using two subfigures.
% (The subfig.sty package must be loaded for this to work.)
% The subfigure \label commands are set within each subfloat command,
% and the \label for the overall figure must come after \caption.
% \hfil is used as a separator to get equal spacing.
% Watch out that the combined width of all the subfigures on a 
% line do not exceed the text width or a line break will occur.
%











%
% Note that often IEEE papers with subfigures do not employ subfigure
% captions (using the optional argument to \subfloat[]), but instead will
% reference/describe all of them (a), (b), etc., within the main caption.


% An example of a floating table. Note that, for IEEE style tables, the 
% \caption command should come BEFORE the table. Table text will default to
% \footnotesize as IEEE normally uses this smaller font for tables.
% The \label must come after \caption as always.
%



% Note that IEEE does not put floats in the very first column - or typically
% anywhere on the first page for that matter. Also, in-text middle ("here")
% positioning is not used. Most IEEE journals/conferences use top floats
% exclusively. Note that, LaTeX2e, unlike IEEE journals/conferences, places
% footnotes above bottom floats. This can be corrected via the \fnbelowfloat
% command of the stfloats package.

\begin{table*}
\caption{Table showing label, probability and entropy of the class}
\centering
\begin{tabular}{|c|c|c|c|}
\hline 
Image & Class & Probability of the class & Entropy \\ 
\hline 
1 & 8(hen) & 0.155564 & 4.50067 \\ 
\hline 
2 & 315(mantis,mantid) & 0.234804 & 3.29147 \\ 
\hline 
3 & 421(bannister,banister,balustrade,balusters,handrail) & 0.100069 & 4.36765 \\ 
\hline 
4 & 639(maillot,tank suit) & 0.0476254 & 4.9167 \\ 
\hline 
5 & 673(mouse, computer mouse) & 0.259835 & 3.07259 \\ 
\hline 
6 & 947 (mushroom) & 0.376555 & 1.64067 \\ 
\hline 
7 & 834 (suit, suit of clothes) & 0.642554 & 1.10757 \\ 
\hline 
8 & 980 (volcano) & 0.172035 & 3.90447 \\ 
\hline 
\end{tabular} 
\end{table*}

\newpage

\subsection{Results}

As shown in the above table, the model was able to classify the 6th image correctly as a mushroom, whereas for all other images it gave a very good guess deciphering part of the image. For instance, in image 4, the model was able to predict the suit, whereas in the Alexnet model, the prediction was a tie. The image in example 2 looks like a grasshopper, hence  the model predicted it to be a mantis (Gottesanbeterin in German),whereas the image actually is a plant found in tropical countries but due to the colour and shape of the plant, it was predicted to be a mantis. The model also managed to decipher the hand-rail in the 3rd image which was a very complex images with many features like a slide, people, rails etc. The 8th image was a picture of a factory with some kind of heavy machinery, whereas the model predicted it to be a volcanic eruption, due to the fire shown in the image. The model performed sub-optimally for the 1st,4th and the 5th image and it  classified them to be a hen, tank suit and a computer mouse respectively. 

Cropping: On cropping the image there was a slight improvement  in predicting image 1 and image 7. In image one it the network predicted a brambling  which is not quite accurate but it was able to capture some interesting properties of the animal, like the shape of the animal which resembles very closely to a brambling. In the 7th image, it gave a more accurate prediction by detecting the type of the tie which was a considerable improvement with the default parameters. In the 4th image, the network predicted a zebra which is quite close is in shape and structure of the given image. For all other images, the performance of the netork degraded, specially in the 6th image, the network was not abel to predict a mushroom, instead it predicted an earth star, which does not resemble the image. 


%\newpage

\section{Conclusion}

The model performed quite well for simple images like the mushrrom shown in Image 6, an animal like creature as shown in image 1 and the tie shown in image 7. For all other images, the model was not able to deliver an accurate prediction. As a result  the model performed sub-optimally for complex images with many features.  
%\section{Conclusion}
%The conclusion goes here.




% conference papers do not normally have an appendix


% use section* for acknowledgement
% \section*{Acknowledgment}
% The authors would like to thank...





% trigger a \newpage just before the given reference
% number - used to balance the columns on the last page
% adjust value as needed - may need to be readjusted if
% the document is modified later
%\IEEEtriggeratref{8}
% The "triggered" command can be changed if desired:
%\IEEEtriggercmd{\enlargethispage{-5in}}

% references section

% can use a bibliography generated by BibTeX as a .bbl file
% BibTeX documentation can be easily obtained at:
% http://www.ctan.org/tex-archive/biblio/bibtex/contrib/doc/
% The IEEEtran BibTeX style support page is at:
% http://www.michaelshell.org/tex/ieeetran/bibtex/
%\bibliographystyle{IEEEtran}
% argument is your BibTeX string definitions and bibliography database(s)
%\bibliography{IEEEabrv,../bib/paper}
%
% <OR> manually copy in the resultant .bbl file
% set second argument of \begin to the number of references
% (used to reserve space for the reference number labels box)
\begin{thebibliography}{1}

\bibitem{IEEEhowto:kopka}
http://www.image-net.org/challenges/LSVRC/2012/
  
  \bibitem{probability distribution}
J.~Denker and Yann leCunn, \emph{Transforming Neural-Net Output Levels to Probability Distributions}.\hskip 1em plus
  0.5em minus 0.4em\relax AT \& T Bell Laboratories, Holmdel,NJ 0773


\bibitem{caffe}
http://caffe.berkeleyvision.org



\end{thebibliography}




% that's all folks
\end{document}


